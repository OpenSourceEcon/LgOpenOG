\documentclass[letterpaper,12pt]{article}
\usepackage{array}
\usepackage{threeparttable}
\usepackage{geometry}
\geometry{letterpaper,tmargin=1in,bmargin=1in,lmargin=0.5in,rmargin=0.5in}
\usepackage{fancyhdr,lastpage}
\pagestyle{fancy}
\lhead{}
\chead{}
\rhead{}
\lfoot{\footnotesize\textsl{OLG, CHP 18}}
\cfoot{}
\rfoot{\footnotesize\textsl{Page \thepage\ of \pageref{LastPage}}}
\renewcommand\headrulewidth{0pt}
\renewcommand\footrulewidth{0pt}
\usepackage[format=hang,font=normalsize,labelfont=bf]{caption}
\usepackage{amsmath}
\usepackage{amssymb}
\usepackage{amsthm}
\usepackage{natbib}
\usepackage{upgreek}
\usepackage{setspace}
\usepackage{float,color}
\usepackage[pdftex]{graphicx}
\usepackage{hyperref}
\usepackage{mathrsfs}
\usepackage{dsfont}
\usepackage{graphicx}
\usepackage{listings}
\hypersetup{colorlinks,linkcolor=red,urlcolor=blue,citecolor=red}
\theoremstyle{definition}
\newtheorem{theorem}{Theorem}
\newtheorem{acknowledgement}[theorem]{Acknowledgement}
\newtheorem{algorithm}[theorem]{Algorithm}
\newtheorem{axiom}[theorem]{Axiom}
\newtheorem{case}[theorem]{Case}
\newtheorem{claim}[theorem]{Claim}
\newtheorem{conclusion}[theorem]{Conclusion}
\newtheorem{condition}[theorem]{Condition}
\newtheorem{conjecture}[theorem]{Conjecture}
\newtheorem{corollary}[theorem]{Corollary}
\newtheorem{criterion}[theorem]{Criterion}
\newtheorem{definition}[theorem]{Definition}
\newtheorem{derivation}{Derivation} % Number derivations on their own
\newtheorem{example}[theorem]{Example}
\newtheorem{exercise}[theorem]{Exercise}
\newtheorem{lemma}[theorem]{Lemma}
\newtheorem{notation}[theorem]{Notation}
\newtheorem{problem}[theorem]{Problem}
\newtheorem{proposition}{Proposition} % Number propositions on their own
\newtheorem{remark}[theorem]{Remark}
\newtheorem{solution}[theorem]{Solution}
\newtheorem{summary}[theorem]{Summary}
%\numberwithin{equation}{section}
\bibliographystyle{aer}
\newcommand\ve{\varepsilon}
\newcommand\cov{\text{Cov}}
\newcommand\var{\text{Var}}
\newcommand\argmax{\text{argmax}}
\newcommand\boldline{\arrayrulewidth{1pt}\hline}

\begin{document}

\title{Chapter 18}
\author{Ruby (Ruishan) Zhang Ian Pitman}
\maketitle

In this chapter, we take model an economy in which two countries that each look like the the $S$-period-lived agent economies from Chapter \ref{Chap_SperSimp}. In contrast to a small open economy model, changes within a country in the large open economy model can affect equilibrium interest rates. We call one country the Home country and the other country the Foreign country, where \textit{Home} and \textit{Foreign} are not relative terms but are fixed country names.

We assume that capital is mobile and that labor is not mobile. That is, Home country household savings can be invested in either the Home country production or in the Foreign country production. However, Home country household labor supply can only be hired by Home country firms. The techniques in this chapter generalize to an $N$-country model.


\section{Households}\label{SecLgOpenHH}


  \subsection{Home household demand for country-specific consumption}\label{SecLgOpenHHhomeSpec}

    In the large open economy model, households value total consumption which is comprised of both Home country output and of Foreign country country output. We define $c_{h,s,t}$ as the total consumption of Home household, age-$s$, period-$t$ as a constant elasticity of substitution (CES) aggregator of Home country age-$s$ household consumption of Home output $c^h_{h,s,t}$ and of Home country age-$s$ household consumption of foreign output $c^f_{h,s,t}$.
    \begin{equation}\label{EqLgOpenHHchCES}
      c_{h,s,t}\equiv \biggl[(1 - \theta_h)^\frac{1}{\ve_h}\bigl(c^h_{h,s,t}\bigr)^\frac{\ve_h-1}{\ve_h} + (\theta_h)^\frac{1}{\ve_h}\bigl(c^f_{h,s,t}\bigr)^\frac{\ve_h-1}{\ve_h}\biggr]^\frac{\ve_h}{\ve_h-1} \quad\forall s,t
    \end{equation}
    The constant elasticity is parameterized by $\ve_h$, the the expenditure share on Foreign goods is given by $\theta_h$. A nice property of the CES aggregator is that it nests the case of perfect substitutes ($\ve=\infty$) in which total Home consumption is just the sum of Home consumption of Home output and Home consumption of Foreign output. It also nests the unit elastic Cobb-Douglas case ($\ve=1$) and the case of perfect compliments ($\ve=0$). However, we will assume that the elasticity of substitution between Home and Foreign consumption is strictly between zero and infinity $\ve\in(0,\infty)$.

    We can solve for the Home household's demand for Home and Foreign consumption, respectively, by solving the expenditure minimization problem subject to a given level of total consumption,
    \begin{equation}\label{EqLgOpenHHminprobH}
      \min_{c^h_{h,s,t},c^f_{h,s,t}} \: P^h_t c^h_{h,s,t} + \frac{P^f_t}{e_t}c^f_{h,s,t} \quad \text{s.t.} \quad c_{h,s,t} \leq \biggl[(1 - \theta_h)^\frac{1}{\ve_h}\bigl(c^h_{h,s,t}\bigr)^\frac{\ve_h-1}{\ve_h} + (\theta_h)^\frac{1}{\ve_h}\bigl(c^f_{h,s,t}\bigr)^\frac{\ve_h-1}{\ve_h}\biggr]^\frac{\ve_h}{\ve_h-1}
    \end{equation}
    where $P^h_t$ is the price of Home country consumption in Home output units, $P^f_t$ is the price of Foreign country consumption in Foreign output units, and $e_t$ is the exchange rate representing the number of Foreign output units per one Home output unit. The Lagrangian for this problem is the following,
    \begin{equation}\label{EqLgOpenHHlagrH}
      \mathcal{L} = P^h_t c^h_{h,s,t} + \frac{P^f_t}{e_t}c^f_{h,s,t} + P_t\left(c_{h,s,t} - \biggl[(1 - \theta_h)^\frac{1}{\ve_h}\bigl(c^h_{h,s,t}\bigr)^\frac{\ve_h-1}{\ve_h} + (\theta_h)^\frac{1}{\ve_h}\bigl(c^f_{h,s,t}\bigr)^\frac{\ve_h-1}{\ve_h}\biggr]^\frac{\ve_h}{\ve_h-1}\right)
    \end{equation}
    where $P_t$ is the multiplier on the constraint and represents the marginal cost of an extra unit of aggregate consumption. So $P_t$ is interpreted as the price of total consumption. The first order conditions are the following:
    \begin{equation}\label{EqLgOpenHHfocChh}
      P^h_t = P_t\left[\frac{(1-\theta_h) c_{h,s,t}}{c^h_{h,s,t}}\right]^{\frac{1}{\ve_h}} \quad\forall s,t
    \end{equation}
    \begin{equation}\label{EqLgOpenHHfocChf}
      \frac{P^f_t}{e_t} = P_t\left[\frac{\theta_h c_{h,s,t}}{c^f_{h,s,t}}\right]^{\frac{1}{\ve_h}} \quad\forall s,t
    \end{equation}
    \begin{equation}\tag{\ref{EqLgOpenHHchCES}}
      c_{h,s,t}= \biggl[(1 - \theta_h)^\frac{1}{\ve_h}\bigl(c^h_{h,s,t}\bigr)^\frac{\ve_h-1}{\ve_h} + (\theta_h)^\frac{1}{\ve_h}\bigl(c^f_{h,s,t}\bigr)^\frac{\ve_h-1}{\ve_h}\biggr]^\frac{\ve_h}{\ve_h-1} \quad\forall s,t
    \end{equation}
    Dividing \eqref{EqLgOpenHHfocChh} by \eqref{EqLgOpenHHfocChf} gives the following relationship.
    \begin{equation}\label{EqLgOpenHHFracExpDomFor}
      \frac{e_t P^h_t}{P^f_t}\left(\frac{c^h_{h,s,t}}{c^f_{h,s,t}}\right)^{\frac{1}{\ve_h}} = \left(\frac{1-\theta_h}{\theta_h}\right)^{\frac{1}{\ve_h}} \quad\forall s,t
    \end{equation}
    Notice that in the Cobb-Douglas case when $\ve_h=1$, the ratio of Home consumption expenditure to Foreign consumption expenditure is a constant. Also, note that solving \eqref{EqLgOpenHHfocChh} and \eqref{EqLgOpenHHfocChf} for $c^h_{h,s,t}$ and $c^f_{h,s,t}$, respectively, gives Home demand equations for consumption of Home goods and Foreign goods.
    \begin{equation}\label{EqLgOpenHHDemChh}
          c^h_{h,s,t} = (1-\theta_h)\left(\frac{P^h_t}{P_t}\right)^{-\ve_h}c_{h,s,t} \quad\forall s,t
    \end{equation}
    \begin{equation}\label{EqLgOpenHHDemChf}
          c^f_{h,s,t} = \theta_h \left(\frac{P^f_t}{e_t P_t}\right)^{-\ve_h}c_{h,s,t} \quad\forall s,t
    \end{equation}

    The expression for the aggregate price index $P_t$ of the Home consumption over Home and Foreign consumption is found by substituting the demand equations \eqref{EqLgOpenHHDemChh} and \eqref{EqLgOpenHHDemChf} into \eqref{EqLgOpenHHchCES}.
    \begin{equation}\label{EqLgOpenHHaggPr}
      P_t = \left[(1-\theta_h)\left(P^h_t\right)^{1-\ve_h} + \theta_h\left(\frac{P^f_t}{e_t}\right)^{1-\ve_h}\right]^{\frac{1}{1-\ve_h}} \quad\forall t
    \end{equation}
    Note that this expression is the Home country consumer price index. In the case of Cobb-Douglas aggregation over Home consumption and Foreign consumption ($\ve_h=1$), the expression for aggregate price is the following.
    \begin{equation}\label{EqLgOpenHHaggPrCobb}
      P_t = \frac{1}{(1-\theta_h)^{1-\theta_h}(\theta_h)^{\theta_h}} \left(P^h_t\right)^{1-\theta_h}\left(\frac{P^f_t}{e_t}\right)^{\theta_h} \quad\forall t
    \end{equation}


  \subsection{Foreign household demand for country-specific consumption}\label{SecLgOpenHHforSpec}

    We define $c_{f,s,t}$ as the total consumption of Foreign household, age-$s$, period-$t$ as a constant elasticity of substitution (CES) aggregator of Foreign country age-$s$ household consumption of Foreign output $c^f_{f,s,t}$ and of Foreign country age-$s$ household consumption of Home output $c^h_{f,s,t}$.
    \begin{equation}\label{EqLgOpenHHcfCES}
      c_{f,s,t}\equiv \biggl[(1 - \theta_f)^\frac{1}{\ve_f}\bigl(c^f_{f,s,t}\bigr)^\frac{\ve_f-1}{\ve_f} + (\theta_f)^\frac{1}{\ve_f}\bigl(c^h_{f,s,t}\bigr)^\frac{\ve_f-1}{\ve_f}\biggr]^\frac{\ve_f}{\ve_f-1} \quad\forall s,t
    \end{equation}
    The constant elasticity is parameterized by $\ve_f$, the the expenditure share on Home goods is given by $\theta_f$.

    We can solve for the Foreign household's demand for Foreign and Home consumption, respectively, by solving the expenditure minimization problem subject to a given level of total consumption.
    \begin{equation}\label{EqLgOpenHHminprobF}
      \min_{c^f_{f,s,t},c^h_{f,s,t}} \: P^f_t c^f_{f,s,t} + e_t P^h_t c^h_{f,s,t} \quad \text{s.t.} \quad c_{f,s,t} \leq \biggl[(1 - \theta_f)^\frac{1}{\ve_f}\bigl(c^f_{f,s,t}\bigr)^\frac{\ve_f-1}{\ve_f} + (\theta_f)^\frac{1}{\ve_f}\bigl(c^h_{f,s,t}\bigr)^\frac{\ve_f-1}{\ve_f}\biggr]^\frac{\ve_f}{\ve_f-1}
    \end{equation}
    The Lagrangian for this problem is the following,
    \begin{equation}\label{EqLgOpenHHlagrF}
      \mathcal{L} = P^f_t c^f_{f,s,t} + e_t P^h_t c^h_{f,s,t} + P^*_t\left(c_{f,s,t} - \biggl[(1 - \theta_f)^\frac{1}{\ve_f}\bigl(c^f_{f,s,t}\bigr)^\frac{\ve_f-1}{\ve_f} + (\theta_f)^\frac{1}{\ve_f}\bigl(c^h_{f,s,t}\bigr)^\frac{\ve_f-1}{\ve_f}\biggr]^\frac{\ve_f}{\ve_f-1}\right)
    \end{equation}
    where $P^*_t$ is the multiplier on the constraint and represents the marginal cost of an extra unit of aggregate consumption. So $P^*_t$ is interpreted as the Foreign price of total consumption. The first order conditions are the following.
    \begin{equation}\label{EqLgOpenHHfocCff}
      P^f_t = P^*_t\left[\frac{(1-\theta_f) c_{f,s,t}}{c^f_{f,s,t}}\right]^{\frac{1}{\ve_f}} \quad\forall s,t
    \end{equation}
    \begin{equation}\label{EqLgOpenHHfocCfh}
      e_t P^h_t = P^*_t\left[\frac{\theta_f c_{f,s,t}}{c^h_{f,s,t}}\right]^{\frac{1}{\ve_f}} \quad\forall s,t
    \end{equation}
    \begin{equation}\tag{\ref{EqLgOpenHHcfCES}}
      c_{f,s,t}= \biggl[(1 - \theta_f)^\frac{1}{\ve_f}\bigl(c^f_{f,s,t}\bigr)^\frac{\ve_f-1}{\ve_f} + (\theta_f)^\frac{1}{\ve_f}\bigl(c^h_{f,s,t}\bigr)^\frac{\ve_f-1}{\ve_f}\biggr]^\frac{\ve_f}{\ve_f-1} \quad\forall s,t
    \end{equation}
    Dividing \eqref{EqLgOpenHHfocCff} by \eqref{EqLgOpenHHfocCfh} gives the following relationship.
    \begin{equation}\label{EqLgOpenHHFracExpForDom}
      \frac{P^f_t}{e_t P^h_t}\left(\frac{c^f_{f,s,t}}{c^h_{f,s,t}}\right)^{\frac{1}{\ve_f}} = \left(\frac{1-\theta_f}{\theta_f}\right)^{\frac{1}{\ve_f}} \quad\forall s,t
    \end{equation}
    Notice that in the Cobb-Douglas case when $\ve_f=1$, the ratio of Home consumption expenditure to Foreign consumption expenditure is a constant. Also, note that solving \eqref{EqLgOpenHHfocCff} and \eqref{EqLgOpenHHfocCfh} for $c^f_{f,s,t}$ and $c^h_{f,s,t}$, respectively, gives Home demand equations for consumption of Home goods and Foreign goods.
    \begin{equation}\label{EqLgOpenHHDemCff}
          c^f_{f,s,t} = (1-\theta_f)\left(\frac{P^f_t}{P^*_t}\right)^{-\ve_f}c_{f,s,t} \quad\forall s,t
    \end{equation}
    \begin{equation}\label{EqLgOpenHHDemCfh}
          c^h_{f,s,t} = \theta_f \left(\frac{e_t P^h_t}{P^*_t}\right)^{-\ve_f}c_{f,s,t} \quad\forall s,t
    \end{equation}

    The expression for the aggregate price index $P^*_t$ of the Foreign consumption over Foreign and Home consumption is found by substituting the demand equations \eqref{EqLgOpenHHDemCff} and \eqref{EqLgOpenHHDemCfh} into \eqref{EqLgOpenHHcfCES}.
    \begin{equation}\label{EqLgOpenHHaggPrF}
      P^*_t = \left[(1-\theta_f)\left(P^f_t\right)^{1-\ve_f} + \theta_f\left(e_t P^h_t\right)^{1-\ve_f}\right]^{\frac{1}{1-\ve_f}} \quad\forall t
    \end{equation}
    Note that this expression is the Foreign country consumer price index. In the case of Cobb-Douglas aggregation over Foreign consumption and Home consumption ($\ve_f=1$), the expression for aggregate price is the following.
    \begin{equation}\label{EqLgOpenHHaggPrCobbF}
      P^*_t = \frac{1}{(1-\theta_f)^{1-\theta_f}(\theta_f)^{\theta_f}} \left(P^f_t\right)^{1-\theta_f}\left(e_t P^h_t\right)^{\theta_f} \quad\forall t
    \end{equation}


  \subsection{Home household's lifetime problem}\label{SecLgOpenHHlifeH}

    A unit measure of Home households is born each period and lives for $S$ periods. These households supply labor inelastically according to the following equation.
    \begin{equation}\label{EqLgOpenHHlabH}
      n_{h,s} =
        \begin{cases}
          1.0 \quad\text{if}\quad s < round\left(\frac{9}{16}\right)S \\
          0.2 \quad\text{if}\quad s \geq round\left(\frac{9}{16}\right)S
        \end{cases}
    \end{equation}

    Because we can specify the Home household's demand for Home consumption \eqref{EqLgOpenHHDemChh} and for Foreign consumption \eqref{EqLgOpenHHDemChf} as a percent of total Home consumption $c_{h,s,t}$, we can simply solve the Home household's problem each period for total Home consumption $c_{h,s,t}$, given Home prices $P^h_t$, Foreign prices $P^f_t$, and the exchange rate $e_t$, as they are summarized by the Home CPI $P_t$ \eqref{EqLgOpenHHaggPr}.

    Let the age-$s$ Home household's budget constraint in each period $t$ be the following,
    \begin{equation}\label{EqLgOpenHHbcH}
      P_t c_{h,s,t} + b_{h,s+1,t+1} = (1 + r_{h,t})b_{h,s,t} + w_{h,t}n_{h,s} \quad\forall s, t
    \end{equation}
    where $r_{h,t}$ is the Home country interest rate (return on savings), and $w_{h,t}$ is the Home country wage.

    \begin{align}
      \max_{\{c_{h,s,t},b_{h,s+1,t+1}\}}\: &\sum_{s=1}^S \beta^{s-1}u\left(c_{h,s,t+s-1}\right) \label{EqLgOpenHHmaxprobH} \\
      &\quad\text{s.t.}\quad P_t c_{h,s,t} + b_{h,s+1,t+1} = (1 + r_{h,t})b_{h,s,t} + w_{h,t}n_{h,s} \tag{\ref{EqLgOpenHHbcH}} \\
      &\quad\text{where}\quad u\left(c_{h,s,t}\right) \equiv \frac{\left(c_{h,s,t}\right)^{1-\sigma} - 1}{1 - \sigma}
    \end{align}

    \begin{equation}\label{EqLgOpenHHeul_bH}
      \frac{P_{t+1}}{P_t}\left(c_{h,s,t}\right)^{-\sigma} = \beta(1 + r_{h,t+1})\left(c_{h,s+1,t+1}\right)^{-\sigma} \quad\text{for}\quad 1\leq s\leq S-1,\quad\text{and}\quad\forall t
    \end{equation}


  \subsection{Foreign household's lifetime problem}\label{SecLgOpenHHlifeF}

    \begin{equation}\label{EqLgOpenHHlabF}
      n_{f,s} =
        \begin{cases}
          1.0 \quad\text{if}\quad s < round\left(\frac{9}{16}\right)S \\
          0.2 \quad\text{if}\quad s \geq round\left(\frac{9}{16}\right)S
        \end{cases}
    \end{equation}

    Let the age-$s$ Foreign household's budget constraint in each period $t$ be the following,
    \begin{equation}\label{EqLgOpenHHbcF}
      P^*_t c_{f,s,t} + b_{f,s+1,t+1} = (1 + r_{f,t})b_{f,s,t} + w_{f,t}n_{f,s} \quad\forall s, t
    \end{equation}
    where $r_{f,t}$ is the Foreign country interest rate (return on savings), and $w_{f,t}$ is the Foreign country wage.

    \begin{align}
      \max_{\{c_{f,s,t},b_{f,s+1,t+1}\}}\: &\sum_{s=1}^S \beta^{s-1}u\left(c_{f,s,t+s-1}\right) \label{EqLgOpenHHmaxprobF} \\
      &\quad\text{s.t.}\quad P^*_t c_{f,s,t} + b_{f,s+1,t+1} = (1 + r_{f,t})b_{f,s,t} + w_{f,t}n_{f,s} \tag{\ref{EqLgOpenHHbcF}} \\
      &\quad\text{where}\quad u\left(c_{f,s,t}\right) \equiv \frac{\left(c_{f,s,t}\right)^{1-\sigma} - 1}{1 - \sigma}
    \end{align}

    \begin{equation}\label{EqLgOpenHHeul_bF}
      \frac{P^*_{t+1}}{P^*_t}\left(c_{f,s,t}\right)^{-\sigma} = \beta(1 + r_{f,t+1})\left(c_{f,s+1,t+1}\right)^{-\sigma} \quad\text{for}\quad 1\leq s\leq S-1,\quad\text{and}\quad\forall t
    \end{equation}


\section{Firms}\label{SecLgOpenFirms}


  \subsection{Home country intermediate goods producers}\label{SecLgOpenFirmIntd_H}

    The Home country is populated by a unit measure of perfectly competitive intermediate goods producers that rent capital from households of both countries and transform those two inputs of household savings into an aggregated capital stock $K_{h,t}$ specific to the Home country. We assume that this is done through a constant elasticity of substitution (CES) production function in which the elasticity of substitution between Home capital and Foreign capital in the Home intermediate goods producer's production function is constant $\phi_h\geq 1$,
    \begin{equation}\label{EqLgOpenFirmKhCES}
      K_{h,t}\equiv \biggl[(1 - \alpha_h)^\frac{1}{\phi_h}\bigl(K^h_{h,t}\bigr)^\frac{\phi_h-1}{\phi_h} + (\alpha_h)^\frac{1}{\phi_h}\bigl(K^f_{h,t}\bigr)^\frac{\phi_h-1}{\phi_h}\biggr]^\frac{\phi_h}{\phi_h-1} \quad\forall t
    \end{equation}
    where $\alpha_h$ is related to the Foreign country capital share in Home country aggregated capital $K_{h,t}$. The two capital inputs on the right-hand-side of \eqref{EqLgOpenFirmKhCES} $K^h_{h,t}$ and $K^f_{h,t}$ represent the total amount of country-specific household savings that was used in the Home country intermediate goods production function.
    \begin{equation}\label{EqLgOpenFirmKhh}
      K^h_{h,t} \equiv \Delta_h \sum_{s=2}^S b_{h,s,t}
    \end{equation}
    \begin{equation}\label{EqLgOpenFirmKhf}
      K^f_{h,t} \equiv (1 - \Delta_f)\sum_{s=2}^S b_{f,s,t}
    \end{equation}
    The variables $\Delta_h$ and $\Delta_f$ are endogenous variables and represent the percent of total Home household savings that goes to Home production and the percent of total Foreign household savings that goes to Foreign production, respectively. The output of the Home country intermediate goods producer is a new type of aggregated capital that can be used by the Home country final goods producer.

    An intuitive analogy for the role these intermediate goods producers fill is a domestic banking sector. The intermediate goods producer in the home country takes household savings from the Home country $K^h_{h,t}$ and household savings from the Foreign country $K^f_{h,t}$ and bundles that savings together as an output for use by the final goods producer described in Section \ref{SecLgOpenFirmFinal_H}. The CES production function describes the technology with which these two different types of capital are combined in the production process and provides demand functions based on the relative price of the two forms of capital.

    We can solve for the Home country intermediate goods producer's demand for Home and Foreign savings, respectively, by solving the expenditure minimization problem subject to a given level of aggregated capital.
    \begin{equation}\label{EqLgOpenFirmMinProbH}
      \min_{K^h_{h,t},K^f_{h,t}} \: r_{h,t} K^h_{h,t} + \frac{r_{f,t}}{e_t}K^f_{h,t} \quad \text{s.t.} \quad K_{h,t} \leq \biggl[(1 - \alpha_h)^\frac{1}{\phi_h}\bigl(K^h_{h,t}\bigr)^\frac{\phi_h-1}{\phi_h} + (\alpha_h)^\frac{1}{\phi_h}\bigl(K^f_{h,t}\bigr)^\frac{\phi_h-1}{\phi_h}\biggr]^\frac{\phi_h}{\phi_h-1}
    \end{equation}
    where $r_{h,t}$ is the interest rate on Home country savings in Home output units from \eqref{EqLgOpenHHbcH}, $r_{f,t}$ is the interest rate on Foreign country savings in Foreign output units from \eqref{EqLgOpenHHbcF}, and $e_t$ is the exchange rate representing the number of Foreign output units per one Home output unit. The Lagrangian for this problem is the following,
    \begin{equation}\label{EqLgOpenFirmlagrH}
      \mathcal{L} = r_{h,t}K^h_{h,t} + \frac{r_{f,t}}{e_t}K^f_{h,t} + r_t\left(K_{h,t} - \biggl[(1 - \alpha_h)^\frac{1}{\phi_h}\bigl(K^h_{h,t}\bigr)^\frac{\phi_h-1}{\phi_h} + (\alpha_h)^\frac{1}{\phi_h}\bigl(K^f_{h,t}\bigr)^\frac{\phi_h-1}{\phi_h}\biggr]^\frac{\phi_h}{\phi_h-1}\right)
    \end{equation}
    where $r_t$ is the multiplier on the constraint and represents the marginal cost of an extra unit of aggregated Home capital. So $r_t$ is interpreted as the price or interest rate of aggregated Home capital. The first order conditions are the following:
    \begin{equation}\label{EqLgOpenFirmfocKhh}
      r_{h,t} = r_t\left[\frac{(1-\alpha_h)K_{h,t}}{K^h_{h,t}}\right]^{\frac{1}{\phi_h}} \quad\forall t
    \end{equation}
    \begin{equation}\label{EqLgOpenFirmfocKhf}
      \frac{r_{f,t}}{e_t} = r_t\left[\frac{\alpha_h K_{h,t}}{K^f_{h,t}}\right]^{\frac{1}{\phi_h}} \quad\forall t
    \end{equation}
    \begin{equation}\tag{\ref{EqLgOpenFirmKhCES}}
      K_{h,t}= \biggl[(1 - \alpha_h)^\frac{1}{\phi_h}\bigl(K^h_{h,t}\bigr)^\frac{\phi_h-1}{\phi_h} + (\alpha_h)^\frac{1}{\phi_h}\bigl(K^f_{h,t}\bigr)^\frac{\phi_h-1}{\phi_h}\biggr]^\frac{\phi_h}{\phi_h-1} \quad\forall t
    \end{equation}
    Dividing \eqref{EqLgOpenFirmfocKhh} by \eqref{EqLgOpenFirmfocKhf} gives the following relationship.
    \begin{equation}\label{EqLgOpenFirmFracCapDomFor}
      \frac{e_t r_{h,t}}{r_{f,t}}\left(\frac{K^h_{h,t}}{K^f_{h,t}}\right)^{\frac{1}{\phi_h}} = \left(\frac{1-\alpha_h}{\alpha_h}\right)^{\frac{1}{\phi_h}} \quad\forall t
    \end{equation}
    Notice that in the Cobb-Douglas case when $\phi_h=1$, the ratio of Home savings expenditure to Foreign savings expenditure is a constant. Also, note that solving \eqref{EqLgOpenFirmfocKhh} and \eqref{EqLgOpenFirmfocKhf} for $K^h_{h,t}$ and $K^f_{h,t}$, respectively, gives Home demand equations for consumption of Home savings and Foreign savings.
    \begin{equation}\label{EqLgOpenFirmDemKhh}
      K^h_{h,t} = (1-\alpha_h)\left(\frac{r_{h,t}}{r_t}\right)^{-\phi_h}K_{h,t} \quad\forall t
    \end{equation}
    \begin{equation}\label{EqLgOpenFirmDemKhf}
      K^f_{h,t} = \alpha_h \left(\frac{r_{f,t}}{e_t r_t}\right)^{-\phi_h}K_{h,t} \quad\forall t
    \end{equation}

    The expression for the interest rate $r_t$ on Home country aggregated capital $K_{h,t}$ is found by substituting the demand equations \eqref{EqLgOpenFirmDemKhh} and \eqref{EqLgOpenFirmDemKhf} into \eqref{EqLgOpenFirmKhCES}.
    \begin{equation}\label{EqLgOpenFirmAggR_h}
      r_t = \left[(1-\alpha_h)\left(r_{h,t}\right)^{1-\phi_h} + \alpha_h\left(\frac{r_{f,t}}{e_t}\right)^{1-\phi_h}\right]^{\frac{1}{1-\phi_h}} \quad\forall t
    \end{equation}
    In the case of Cobb-Douglas aggregation over Home household savings and Foreign household savings ($\phi_h=1$), the expression for the interest rate on Home aggregated capital is the following.
    \begin{equation}\label{EqLgOpenFirmAggR_h_Cobb}
      r_t = \frac{1}{(1-\alpha_h)^{1-\alpha_h}(\alpha_h)^{\alpha_h}} \left(r_{h,t}\right)^{1-\alpha_h}\left(\frac{r_{f,t}}{e_t}\right)^{\alpha_h} \quad\forall t
    \end{equation}


  \subsection{Foreign country intermediate goods producers}\label{SecLgOpenFirmIntd_F}

    The Foreign country is also populated by a unit measure of perfectly competitive intermediate goods producers that rent capital from households of both countries and transform those two inputs of household savings into an aggregated capital stock $K_{f,t}$ specific to the Foreign country. We assume that this is done through a constant elasticity of substitution (CES) production function in which the elasticity of substitution between Home capital and Foreign capital in the Foreign intermediate goods producer's production function is constant $\phi_f\geq 1$,
    \begin{equation}\label{EqLgOpenFirmKfCES}
      K_{f,t}\equiv \biggl[(1 - \alpha_f)^\frac{1}{\phi_f}\bigl(K^f_{f,t}\bigr)^\frac{\phi_f-1}{\phi_f} + (\alpha_f)^\frac{1}{\phi_f}\bigl(K^h_{f,t}\bigr)^\frac{\phi_f-1}{\phi_f}\biggr]^\frac{\phi_f}{\phi_f-1} \quad\forall t
    \end{equation}
    where $\alpha_f$ is related to the Home country capital share in Foreign country aggregated capital $K_{f,t}$. The two capital inputs on the right-hand-side of \eqref{EqLgOpenFirmKfCES} $K^f_{f,t}$ and $K^h_{f,t}$ represent the total amount of country-specific household savings that was used in the Foreign country intermediate goods production function.
    \begin{equation}\label{EqLgOpenFirmKff}
      K^f_{f,t} \equiv \Delta_f \sum_{s=2}^S b_{f,s,t}
    \end{equation}
    \begin{equation}\label{EqLgOpenFirmKfh}
      K^h_{f,t} \equiv (1 - \Delta_h)\sum_{s=2}^S b_{h,s,t}
    \end{equation}
    The variables $\Delta_f$ and $\Delta_h$ are endogenous variables and represent the percent of total Foreign household savings that goes to Foreign production and the percent of total Home household savings that goes to Home production, respectively. The output of the Foreign country intermediate goods producer is a new type of aggregated capital that can be used by the Foreign country final goods producer.

    We can solve for the Foreign country intermediate goods producer's demand for Foreign and Home savings, respectively, by solving the expenditure minimization problem subject to a given level of aggregated capital.
    \begin{equation}\label{EqLgOpenFirmMinProbF}
      \min_{K^f_{f,t},K^h_{f,t}} \: r_{f,t} K^f_{f,t} + e_t r_{h,t}K^h_{f,t} \quad \text{s.t.} \quad K_{f,t} \leq \biggl[(1 - \alpha_f)^\frac{1}{\phi_f}\bigl(K^f_{f,t}\bigr)^\frac{\phi_f-1}{\phi_f} + (\alpha_f)^\frac{1}{\phi_f}\bigl(K^h_{f,t}\bigr)^\frac{\phi_f-1}{\phi_f}\biggr]^\frac{\phi_f}{\phi_f-1}
    \end{equation}
    The Lagrangian for this problem is the following,
    \begin{equation}\label{EqLgOpenFirmlagrF}
      \mathcal{L} = r_{f,t}K^f_{f,t} + e_t r_{h,t}K^h_{f,t} + r^*_t\left(K_{f,t} - \biggl[(1 - \alpha_f)^\frac{1}{\phi_f}\bigl(K^f_{f,t}\bigr)^\frac{\phi_f-1}{\phi_f} + (\alpha_f)^\frac{1}{\phi_f}\bigl(K^h_{f,t}\bigr)^\frac{\phi_f-1}{\phi_f}\biggr]^\frac{\phi_f}{\phi_f-1}\right)
    \end{equation}
    where $r^*_t$ is the multiplier on the constraint and represents the marginal cost of an extra unit of aggregated Foreign capital. So $r^*_t$ is interpreted as the price or interest rate of aggregated Foreign capital. The first order conditions are the following:
    \begin{equation}\label{EqLgOpenFirmfocKff}
      r_{f,t} = r^*_t\left[\frac{(1-\alpha_f)K_{f,t}}{K^f_{f,t}}\right]^{\frac{1}{\phi_f}} \quad\forall t
    \end{equation}
    \begin{equation}\label{EqLgOpenFirmfocKfh}
      e_t r_{h,t} = r^*_t\left[\frac{\alpha_f K_{f,t}}{K^h_{f,t}}\right]^{\frac{1}{\phi_f}} \quad\forall t
    \end{equation}
    \begin{equation}\tag{\ref{EqLgOpenFirmKfCES}}
      K_{f,t}= \biggl[(1 - \alpha_f)^\frac{1}{\phi_f}\bigl(K^f_{f,t}\bigr)^\frac{\phi_f-1}{\phi_f} + (\alpha_f)^\frac{1}{\phi_f}\bigl(K^h_{f,t}\bigr)^\frac{\phi_f-1}{\phi_f}\biggr]^\frac{\phi_f}{\phi_f-1} \quad\forall t
    \end{equation}
    Dividing \eqref{EqLgOpenFirmfocKff} by \eqref{EqLgOpenFirmfocKfh} gives the following relationship.
    \begin{equation}\label{EqLgOpenFirmFracCapForDom}
      \frac{r_{f,t}}{e_t r_{h,t}}\left(\frac{K^f_{f,t}}{K^h_{f,t}}\right)^{\frac{1}{\phi_f}} = \left(\frac{1-\alpha_f}{\alpha_f}\right)^{\frac{1}{\phi_f}} \quad\forall t
    \end{equation}
    Notice that in the Cobb-Douglas case when $\phi_f=1$, the ratio of Foreign savings expenditure to Home savings expenditure is a constant. Also, note that solving \eqref{EqLgOpenFirmfocKff} and \eqref{EqLgOpenFirmfocKfh} for $K^f_{f,t}$ and $K^h_{f,t}$, respectively, gives Foreign demand equations for consumption of Foreign savings and Home savings.
    \begin{equation}\label{EqLgOpenFirmDemKff}
      K^f_{f,t} = (1-\alpha_f)\left(\frac{r_{f,t}}{r^*_t}\right)^{-\phi_f}K_{f,t} \quad\forall t
    \end{equation}
    \begin{equation}\label{EqLgOpenFirmDemKfh}
      K^h_{f,t} = \alpha_f \left(\frac{e_t r_{h,t}}{r^*_t}\right)^{-\phi_f}K_{f,t} \quad\forall t
    \end{equation}

    The expression for the interest rate $r^*_t$ on Foreign country aggregated capital $K_{f,t}$ is found by substituting the demand equations \eqref{EqLgOpenFirmDemKff} and \eqref{EqLgOpenFirmDemKfh} into \eqref{EqLgOpenFirmKfCES}.
    \begin{equation}\label{EqLgOpenFirmAggR_f}
      r^*_t = \left[(1-\alpha_f)\left(r_{f,t}\right)^{1-\phi_f} + \alpha_f \left(e_t r_{h,t}\right)^{1-\phi_f}\right]^{\frac{1}{1-\phi_f}} \quad\forall t
    \end{equation}
    In the case of Cobb-Douglas aggregation over Foreign household savings and Home household savings ($\phi_f=1$), the expression for the interest rate on Foreign aggregated capital is the following.
    \begin{equation}\label{EqLgOpenFirmAggR_f_Cobb}
      r^*_t = \frac{1}{(1-\alpha_f)^{1-\alpha_f}(\alpha_f)^{\alpha_f}} \left(r_{f,t}\right)^{1-\alpha_f}\left(e_t r_{h,t}\right)^{\alpha_f} \quad\forall t
    \end{equation}


  \subsection{Home country final goods producers}\label{SecLgOpenFirmFinal_H}

    A unit measure of perfectly competitive final goods producing firms produce final goods output in the Home country $Y_{h,t}$ by renting aggregated capital $K_{h,t}$ from the representative intermediate goods firm and hiring labor $L_{h,t}$
    \begin{equation}\label{EqLgOpenProdFunc_H}
      Y_{h,t} = Z_h \left(K_{h,t}\right)^{\gamma_h}\left(L_{h,t}\right)^{1-\gamma_h} \quad\forall t
    \end{equation}
    where $Z_h$ is Home-country total factor productivity, $K_{h,t}$ is Home country aggregated capital demand in period $t$, and $L_{h,t}$ is Home country labor demand in period $t$.

    The profit function of the representative Home country firm is the following,
    \begin{equation}\label{EqLgOpenProfit_H}
      Pr_{h,t} = P^h_t Z_h \left(K_{h,t}\right)^{\gamma_h}\left(L_{h,t}\right)^{1-\gamma_h} - w_{h,t}L_{h,t} - (r_t + \delta_h)K_{h,t} \quad\forall t
    \end{equation}
    where $\delta_h$ is the depreciation rate of aggregated capital in the Home country. The two first order conditions for the representative Home country firm are the following.
    \begin{align}
      r_t &= \gamma_h P^h_t Z_h\left(\frac{L_{h,t}}{K_{h,t}}\right)^{1-\gamma_h} - \delta_h \label{EqLgOpenFirmFOCK_H} \\
      w_{h,t} &= (1-\gamma_h)P^h_t Z_h\left(\frac{K_{h,t}}{L_{h,t}}\right)^{\gamma_h} \label{EqLgOpenFirmFOCL_H}
    \end{align}


  \subsection{Foreign country final goods producers}\label{SecLgOpenFirmFinal_F}

    A unit measure of perfectly competitive final goods producing firms also produce output in the Foreign country $Y_{f,t}$ by renting aggregated capital $K_{f,t}$ from the representative intermediate goods firm and hiring labor $L_{f,t}$
    \begin{equation}\label{EqLgOpenProdFunc_F}
      Y_{f,t} = Z_f \left(K_{f,t}\right)^{\gamma_f}\left(L_{f,t}\right)^{1-\gamma_f} \quad\forall t
    \end{equation}
    where $Z_f$ is Foreign-country total factor productivity, $K_{f,t}$ is Foreign country aggregated capital demand in period $t$, and $L_{f,t}$ is Foreign country labor demand in period $t$.

    The profit function of the representative Foreign country firm is the following,
    \begin{equation}\label{EqLgOpenProfit_F}
      Pr_{f,t} = P^f_t Z_f \left(K_{f,t}\right)^{\gamma_f}\left(L_{f,t}\right)^{1-\gamma_f} - w_{f,t}L_{f,t} - (r^*_t + \delta_f)K_{f,t} \quad\forall t
    \end{equation}
    where $\delta_f$ is the depreciation rate in the Foreign country. The two first order conditions for the representative Foreign country firm are the following.
    \begin{align}
      r^*_t &= \gamma_f P^f_t Z_f\left(\frac{L_{f,t}}{K_{f,t}}\right)^{1-\gamma_f} - \delta_f \label{EqLgOpenFirmFOCK_F} \\
      w_{f,t} &= (1-\gamma_f)P^f_t Z_f\left(\frac{K_{f,t}}{L_{f,t}}\right)^{\gamma_f} \label{EqLgOpenFirmFOCL_F}
    \end{align}


\section{Market Clearing}\label{SecLgOpenMC}

  \begin{equation}\label{EqLgOpenMC_Lh}
    L_{h,t} = \sum_{s=1}^S n_{h,s} \quad\forall t
  \end{equation}
  \begin{equation}\label{EqLgOpenMC_Lf}
    L_{f,t} = \sum_{s=1}^S n_{f,s} \quad\forall t
  \end{equation}
  \begin{equation}\label{EqLgOpenMC_Kh}
    K^h_{h,t} + K^h_{f,t} = \sum_{s=2}^S b_{h,s,t} \quad\forall t
  \end{equation}
  \begin{equation}\label{EqLgOpenMC_Kf}
    K^f_{f,t} + K^f_{h,t} = \sum_{s=2}^S b_{f,s,t} \quad\forall t
  \end{equation}

  \textcolor{red}{I am not sure about these two resource constraints, but I think they are redundant by Walra's Law. In particular, they should be something like $Y = C + I + NX$. But my intuition is that net exports have to be zero in this model because the value of imports must equal the value of exports.}
  \begin{equation}\label{EqLgOpenMC_Yh}
    Y_{h,t} = C^h_{h,t} + C^h_{f,t} + K^h_{h,t+1} - (1 - \delta_h)K^h_{h,t} + K^h_{f,t+1}- (1 - \delta_f)K^h_{f,t}\quad\forall t
  \end{equation}
  \begin{equation}\label{EqLgOpenMC_Yf}
    Y_{f,t} = C^f_{f,t} + C^f_{h,t} + K^f_{f,t+1} - (1 - \delta_f)K^f_{f,t} + K^f_{h,t+1}- (1 - \delta_h)K^f_{h,t} \quad\forall t
  \end{equation}

  This final market clearing equation says that the exchange rate makes the value of exports from the Home country to the foreign country equal to the value of exports from the Foreign country to the Home country. This condition makes net exports in both countries equal to zero (balanced trade).
  \begin{equation}\label{EqLgOpenMC_et}
    e_t P^h_t\sum_{s=1}^S c^h_{f,s,t} = P^f_t\sum_{s=1}^S c^f_{h,s,t} \quad\forall t
  \end{equation}

  The final equation is an interest rate parity condition. The return to household savings in both countries must be equivalent in order for the household to be indifferent between investing in the Home country intermediate goods production and in the Foreign country intermediate goods production.
  \begin{equation}\label{EqLgOpenMC_rt}
    e_t(1 + r_{h,t}) = (1 + r_{f,t}) \quad\forall t
  \end{equation}
  This equation means that one unit of Home household savings invested in the Home production process earns a return of $1 + r_{h,t+1}$ tomorrow. One unit of Home household savings invested in the Foreign production process earns the return of $1 + r_{f,t+1}$ tomorrow, which must be converted back into Home country units by dividing by tomorrow's exchange rate.


\section{Solution Method}\label{SecLgOpenSol}


  \section{Steady-state solution method}\label{SecLgOpenSolSS}

    \begin{enumerate}
      \item Guess values for $\bar{P}^h$, $\bar{P}^f$, $\bar{r}^h$, and $\bar{r}^f$.
      \begin{itemize}
        \item Get $\bar{e}$ immediately from \eqref{EqLgOpenMC_rt}
        \item Get $\bar{r}$ and $\bar{r}^*$ from \eqref{EqLgOpenFirmAggR_h_Cobb} and \eqref{EqLgOpenFirmAggR_f_Cobb}, respectively
        \item These imply a value for $\bar{w}_h$ if the Home country production function is Cobb-Douglas ($\ve_h=1$) by combining \eqref{EqLgOpenFirmFOCK_H} and \eqref{EqLgOpenFirmFOCL_H}. Otherwise, you have to also make a guess for $\bar{w}_h$ in the outer loop.
        \item These imply a value for $\bar{w}_f$ if the Foreign country production function is Cobb-Douglas ($\ve_f=1$) by combining \eqref{EqLgOpenFirmFOCK_F} and \eqref{EqLgOpenFirmFOCL_F}. Otherwise, you have to also make a guess for $\bar{w}_f$ in the outer loop.
        \item These imply a value for $\bar{P}$ through \eqref{EqLgOpenHHaggPr}.
        \item These imply a value for $\bar{P}^*$ through \eqref{EqLgOpenHHaggPrF}.
      \end{itemize}
      \item Given $\left(\bar{P}^h, \bar{P}^f, \bar{e}, \bar{P}, \bar{P}^*, \bar{r}^h, \bar{r}^f, \bar{w}_h, \bar{w}_f\right)$, solve each household's lifetime decisions in each country $\{\bar{c}_{h,s},\bar{b}_{h,s+1}\}_{s=1}^S$ and $\{\bar{c}_{f,s},\bar{b}_{f,s+1}\}_{s=1}^S$ using $S-1$ Euler equations from the Home country \eqref{EqLgOpenHHeul_bH} and $S-1$ Euler equations from the Foreign country \eqref{EqLgOpenHHeul_bF}.
      \item Solve for implied values of $\left(\bar{P}^h, \bar{P}^f, \bar{r}^h, \bar{r}^f\right)$ from household optimized values $\{\bar{c}_{h,s},\bar{b}_{h,s+1}\}_{s=1}^S$ and $\{\bar{c}_{f,s},\bar{b}_{f,s+1}\}_{s=1}^S$ by first writing all necessary values in terms of $\left(\bar{P}^h, \bar{P}^f, \bar{r}^h, \bar{r}^f\right)$.
      \begin{itemize}
        \item $\bar{L}_h$ and $\bar{L}_f$ are essentially exogenous through \eqref{EqLgOpenMC_Lh} and \eqref{EqLgOpenMC_Lf}.
        \item $\bar{e}$ can be written in terms of $\bar{r}_f$ and $\bar{r}_h$ using \eqref{EqLgOpenMC_rt}.
        \item $\bar{P}$ and $\bar{P}^*$ can be found using the price aggregator equations \eqref{EqLgOpenHHaggPr} and \eqref{EqLgOpenHHaggPrF}.
        \item Find $\bar{c}_{h,s}^h, \bar{c}_{h,s}^f, \bar{c}_{f,s}^f, \bar{c}_{f,s}^h$ using equations \eqref{EqLgOpenHHDemChh}, \eqref{EqLgOpenHHDemChf}, \eqref{EqLgOpenHHDemCff}, \eqref{EqLgOpenHHDemCfh} since we know aggregate consumption.
        \item Use the capital market clearing conditions, \eqref{EqLgOpenMC_Kh} and \eqref{EqLgOpenMC_Kh}, and firm capital demand ratios,  \eqref{EqLgOpenFirmFracCapForDom} and \eqref{EqLgOpenFirmFracCapDomFor}, to find $\bar{K}_h^h, \bar{K}_h^f, \bar{K}_f^f, \bar{K}_f^h$ in terms of the four variables.
        \item We can use intermediate capital to write $\bar{K}_h, \bar{K}_f, \bar{Y}_h, \bar{Y}_f$ in terms of the four variables using \eqref{EqLgOpenFirmKhCES}, \eqref{EqLgOpenFirmKfCES}, \eqref{EqLgOpenProdFunc_H}, and \eqref{EqLgOpenProdFunc_F} respectively.
        \item $\bar{r}$ and $\bar{r}^*$ can be found using the capital interest aggregator equations \eqref{EqLgOpenFirmAggR_h} and \eqref{EqLgOpenFirmAggR_f}.
      \end{itemize}
      \item The equations that will be used as the four Euler Errors are:
      \begin{itemize}
        \item Resource constraints: \eqref{EqLgOpenMC_Yh} and \eqref{EqLgOpenMC_Yf}.
        \item Interest rates from firm's problem: \eqref{EqLgOpenFirmFOCK_H} and \eqref{EqLgOpenFirmFOCK_F}.
        \item Note that equation \eqref{EqLgOpenMC_et} is extra (redundant Walras' equation?).
      \end{itemize}
      \item Iterate until find fixed point.
    \end{enumerate}

\end{document}
